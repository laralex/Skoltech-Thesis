\chapter{Conclusions}\label{chapter:conclusions}

In this thesis, we firstly implemented an Augmented Reality application for Android mobile devices. It allows to detect surfaces in the camera view and place full-body human avatars on them. The correct images of the avatars are synthesized by a DNN, which was specifically integrated to work natively on the mobile device hardware, in real-time. The integration includes inference of a human body 3D model for a certain body pose, and projection of it as an image with of neural network's inputs. By means of quantization, data layout and parallel GPU processing we arranged an efficient way of generating the input data that can be directly forwarded to the DNN processing, without additional memory reordering required. Using Qualcomm Snapdragon hardware capabilities with Digital Signal Processor computing unit, we achieved real-time DNN inference with latency well above minimum requirement and sufficient resolution of synthesized images. Even bigger resolutions are available, still within real-time performance. We propose to use an algorithmic approach of dynamic cropping of the physical camera's view frustum, to always render input images with avatar being at the center and with only the body parts that will be visible in the current view of AR. This lead to automatically better quality in far-away view on the avatar.

Secondly, we carried our research on improving the training procedure of the DNN, to improve quality of images that can be encountered in AR scenario. Using the nature of the DNN's input data, we propose to use camera-space augmentations to train the neural network on zoomed images. We study how such training can improve stability of synthesized images to minor shifts in the viewpoint, but also reducing original visual quality of full-body images generated by the baseline. We tried to restore the full-body quality back by using well-known regularization approaches, as well as our own insights. However, the effect of the researched approaches seems ever so slight. Although we had a few successful experiments that solve the issues, they come at a cost of being less efficiently implemented on the mobile hardware, leading to drastic decrease in real-time performance. We conclude this work, pointing out that .

% relevance

% take home message

