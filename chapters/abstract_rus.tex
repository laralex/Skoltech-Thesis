\emergencystretch 3em
\par
При разработке моделей искусственного интеллекта, наибольшее внимание уделяется качеству (точности) результатов. Однако, некоторые модели предполагается выполнять на ограниченном пользовательском аппаратном обеспечении. Поэтому, для поддержания высокой производительности модели, необходимо учитывать затрачиваемые вычислительные ресурсы.

\par
В данной магистерской диссертации описывается реализация прикладного приложения для мобильных устройств, со встроенной глубокой нейронной сетью. Приложение визуализирует изображения человеческого 3D аватара, генерируемые нейронной сетью. Генерация изображений происходит в реальном времени, более 30 кадров в секунду. Изображения аватаров учитывают пространственное положение устройства, и при визуализации поверх изображений с камеры, образуется эффект дополненной реальности (Augmented Reality, AR). Пользователь может осматривать аватар с разных углов, в отдалении и приближении. Следовательно, изображения аватаров должны сохранять качество с углов обзора, отсутствующих в тренировочных данных. Основной вклад данной работы заключается в организации вычислений для достижения производительности в реальном времени. Дополнительно, были исследованы пути повышения визуального качества изображений в приложении с произвольных точек обзора.	

\par
В результате, в мобильном приложении baseline модель \alert{\cite{stylepeople}} может вычисляться в реальном времени, вплоть до разрешения $640\times640$ пикселей. Для повышения визуального качества аватаров, нейронная сеть получает и выдает данные так, чтобы видимая часть аватара занимала как можно больше места в кадре. Подход обучения исходной нейронной сети был дополнен, для обучения в масштабах изображений и в полный рост, и в крупном плане, без возникновения визуальных артифактов. 

\par
\textbf{Ключевые слова:} Вычисления в реальном времени, Мобильное аппаратное обеспечение, Нейронный рендеринг, Глубинная нейронная сеть, Реконструкция аватаров людей.  
