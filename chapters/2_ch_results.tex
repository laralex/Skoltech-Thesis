\chapter{Results and Discussion}\label{chapter:results}

\section{Mobile performance}\label{results:performance}
\section{Quantitative analysis}\label{results:metrics}
\section{Qualitative analysis}\label{results:quality}

%\begin{figure}[!htbp]
%	\centering
%	\adjustbox{minipage=1.3em}{\subcaption{}}%
%	\begin{subfigure}[t]{\dimexpr.5\textwidth-1.5em\relax}		\includegraphics[height=5cm]{\imgpath/perlin-terrain}
%	\end{subfigure}
%	%\hfill %выровнять
%	\adjustbox{minipage=1.3em}{\subcaption{}\label{fig:fractal-tree}}%
%	\begin{subfigure}[t]{\dimexpr.5\linewidth-1.5em\relax}
%		\includegraphics[height=5cm]{\imgpath/fractal-tree}
%	\end{subfigure}%
%	
%	\captionsetup{justification=centering} %центрировать
%	\caption{Примеры применения известных функций и фракталов, в том числе: {\itshape a} --- ландшафт из карты высот с шумом Перлина; {\itshape b} --- фрактальная модель дерева.} \label{fig:functions-and-fractals}  
%\end{figure}

%\begin{figure}[!htbp]
%	\captionsetup[subfigure]{justification=centering}
%	\centering
%	\begin{subfigure}[b]{0.24\textwidth}%{\dimexpr.25\textwidth-1em\relax}		
%		\centering
%		\includegraphics[height=5cm]{\imgpath/building-from-image-1} %height=4.8cm
%		\subcaption{}\label{fig:building-from-image:input}
%	\end{subfigure}
%	\hskip -2ex
%	\begin{subfigure}[b]{0.24\textwidth}%{\dimexpr.25\linewidth-1em\relax}
%		\centering
%		\includegraphics[height=5cm]{\imgpath/building-from-image-2}
%		\subcaption{}\label{fig:building-from-image:side-detection}
%	\end{subfigure}
%	\hskip -6ex
%	\begin{subfigure}[b]{0.24\textwidth}%{\dimexpr.25\linewidth-1em\relax}
%		\centering
%		\includegraphics[height=5cm]{\imgpath/building-from-image-3}
%		\subcaption{}\label{fig:building-from-image:simplification}
%	\end{subfigure}
%	\begin{subfigure}[b]{0.24\textwidth}%{\dimexpr.25\linewidth-1em\relax}
%		\centering
%		\includegraphics[height=5cm]{\imgpath/building-from-image-4}
%		\subcaption{}\label{fig:building-from-image:output}
%	\end{subfigure}
%	
%	\captionsetup{justification=centering} %центрировать
%	\caption{Промежуточные результаты алгоритма построения грамматики здания из реального изображения \cite{nishida-buidling-from-image-2018}: 
%		{\itshape a} --- входное изображение; 
%		{\itshape b} --- распознанная сторона здания;
%		{\itshape c} --- результат снижения сложности данных;
%		{\itshape d} --- готовая модель.
%	} \label{fig:building-from-image}  
%\end{figure}

%\begin{table} [htbp]% Пример записи таблицы с номером, но без отображаемого наименования
%	\centering
%	\caption{Сравнение возможностей некоторых существующих решений}%
%	\label{tab:other-solutions}		
%%	\begin{SingleSpace}
%		%		\resizebox{1\linewidth}{!}{
%			\begin{tabular}{|Q{0.13}|Q{0.09}|Z{0.11}|Q{0.23}|Z{0.13}|Q{0.15}|}\hline
%				\textbf{Название системы} & 
%				\textbf{Тип} & 
%				\textbf{Целевой контент} & 
%				\textbf{Усилия для получения результата} & 
%				\textbf{Возможн. экспорта (от 0 до 5)} & 
%				\textbf{Генерация текстур} \\ \hline
%				
%				SceneCity & плагин Blender & модели городов & 
%				Ручная интуитивная настройка &
%				2 & 
%				\multicolumn{1}{c|}{-} \\ \hline
%				
%				Procedural Buildings & модуль Unreal Engine & модели зданий &
%				Ручная настройка грамматики и ручная калибровка & 
%				0 &
%				\multicolumn{1}{c|}{-} \\ \hline
%				
%				Fast Architecture & плагин 3ds Max & модели зданий & 
%				Простая инициализация численных параметров &
%				5 &
%				\multicolumn{1}{c|}{-} \\ \hline
%				
%				Maya Structures & плагин Maya & модели зданий &
%				Простая инициализация базовых компонентов &
%				5 &
%				Процедурное комбинирование \\ \hline
%				
%				Esri CityEngine & ПО & модели городов &
%				Простое прототипирование формы зданий &  
%				5 &
%				Собственная база текстур \\ \hline
%				
%				GameSim Procedural Modeling & ПО & модели городов &
%				Сложная настройка числовых параметров &
%				4 &
%				Собственная база текстур \\ \hline		
%			\end{tabular}	
%			%		}
%%	\end{SingleSpace}
%\end{table} 

%\begin{algorithm} %[h]
%	\SetKwProg{myfn}{Function}{}{} %write in 2nd agrument <<Algorithm>>, <<Procedure>> etc
%	\SetKwProg{myalg}{Algorithm}{}{} %write in 2nd agrument <<Algorithm>>, <<Procedure>> etc
%	\SetKwFunction{DeriveFromGrammar}{DeriveFromGrammar} 
%	\nonl\myfn{\DeriveFromGrammar}{
%		\KwInput{нетерминальные символы $N$,
%			терминальные символы $\Sigma$, 
%			начальное слово $S$, 
%			грамматич. правила над негеометр. символами $R$,
%			грамматич. правила над геометр. символами $\hat{R}$,
%			лимит рекурсии $depth$
%		} 
%		\KwOutput{итоговое слово из символов грамматики $\overbar{S} = (\Sigma\cup N)^{*}$}
%		\If{$depth = 0$}{\Return $S$\;}	
%		$\hat{R} \leftarrow \hat{R} \cup \{SplitRule, TransformRule\}$\;
%		\For (\tcp*[h]{$R_A$ -- символы текущего слова}){$\forall s_i \in S$\label{step:word-iteration}}{
%			\eIf{$IsGeometrical(s_i)$}{
%				$R_A \leftarrow R \cup \hat{R}$\;
%			}{
%				$R_A \leftarrow R$\;
%			}
%			\For(\tcp*[h]{$R_A$ -- применимые правила для символа}) {$\forall r_j \in R_A$} {
%				\If{$Antedecent(r_j) = s_i$}{
%					$S \leftarrow Replace(S, s_i, Consequence(r_j))$\;
%					$DeriveFromGrammar(N, \Sigma, S, R, \hat{R}, depth - 1)$\;	
%				}	
%			}
%		}
%		\Return $S$;
%	}
%	\caption{Псевдокод использования описанной грамматики для вывода итогового слова}\label{alg:grammar-derivation}
%	\index[ru]{алгоритм!\texttt{DiagnosticTestsScaling\-AndInferring}} %нужен ручной перенос \- из-за ошибки в MakeIndex для команды \texttt
%	%ключевые слова <<алгоритм>> и <<algorithm>> не менять
%	\index[en]{algorithm!\texttt{DiagnosticTestsScaling\-AndInferring}} %нужен ручной перенос \- из-за ошибки в MakeIndex для команды \texttt
%\end{algorithm} 

%\begin{figure}[!htbp]
%	\captionsetup[subfigure]{justification=centering}
%	\centering
%	\begin{subfigure}[b]{0.33\textwidth}%{\dimexpr.25\textwidth-1em\relax}	
%		\centering
%		\includegraphics[height=5cm]{\imgpath/e-init-split} %height=4.8cm
%		\subcaption{}\label{fig:generation:init}
%	\end{subfigure}
%	%\hskip -2ex
%	\begin{subfigure}[b]{0.33\textwidth}%{\dimexpr.25\linewidth-1em\relax}
%		\centering
%		\includegraphics[height=5cm]{\imgpath/e-roof-vertical}
%		\subcaption{}\label{fig:generation:subdivide}
%	\end{subfigure}
%	\hskip -1ex
%	\begin{subfigure}[b]{0.33\textwidth}%{\dimexpr.25\linewidth-1em\relax}
%		\centering
%		\includegraphics[height=5cm]{\imgpath/e-end}
%		\subcaption{}\label{fig:generation:output}
%	\end{subfigure}
%	
%	\captionsetup{justification=centering} %центрировать
%	\caption{Вариант преобразования слова по грамматике и пример модели, которая может быть построена на каждом этапе: 
%		{\itshape a} --- начальный этап - объем, разделенный на горизонтальные уровни; 
%		{\itshape b} --- некоторые уровни подразделяются на сегменты, а самый верхний уровень преобразован в символ крыши;
%		{\itshape c} --- возможный результат визуализации выведенного слова.
%	} \label{fig:generation}  
%\end{figure}

%style=nonumbers,
%\begin{figure}[!htbp]
%	\centering
%	\begin{lstlisting}[
%		language=XML,
%		numbers=none,
%		caption=,
%		basicstyle=\fontsize{12}{14}\selectfont\ttfamily,
%		morekeywords={encoding,models,group,name,src,type,model},
%		]
%		<?xml version="1.0" encoding="UTF-8"?>
%		<models>
%		<group name="Windows">
%		<model name="2 sections window" src="windows/window_1.obj" type="obj"/>
%		<model name="Plastic window" src="windows/window_2.obj" type="obj"/>
%		<model name="Dull arc window" src="windows/window_3.obj" type="obj"/>
%		<model name="Sliding window" src="windows/window_4.obj" type="obj"/>
%		</group>
%		<group name="Doors">
%		<model name="Decorated wooden door" src="doors/door_1.obj" type="obj"/>
%		<model name="Classic wooden door" src="doors/door_2.obj" type="obj"/>
%		<model name="Big arc door" src="doors/door_3.obj" type="obj"/>
%		<model name="2 section plastic door" src="doors/door_4.obj" type="obj"/>
%		</group>
%		</models>
%	\end{lstlisting}
%%	\captionsetup{justification=centering}
%	\caption{Example}\label{code:manifest}
%\end{figure}
%
%\begin{figure}[!htbp]
%	\centering
%	\begin{lstlisting}[
%		numbers=none,
%		basicstyle=\fontsize{12}{14}\selectfont\ttfamily,
%		caption=,
%		]
%		public interface IVisualizer
%		{
%			string GetDescription();
%			void VisualizeModel(Stream model, ModelMetaBase modelMeta, 
%			Stream materialLibrary, Stream[] materialFiles);
%			void Shutdown();
%		} 
%	\end{lstlisting}
%%	\captionsetup{justification=centering}
%	\caption{Example}\label{code:visualizer-charp-interface}
%\end{figure}