%% This is an example first chapter.  You should put chapter/appendix that you
%% write into a separate file, and add a line \include{yourfilename} to
%% main.tex, where `yourfilename.tex' is the name of the chapter/appendix file.
%% You can process specific files by typing their names in at the 
%% \files=
%% prompt when you run the file main.tex through LaTeX.
\chapter{Introduction}

Computer Graphics is a vast discipline of Computer Sciences. In general, it studies theoretical methods for synthesis of 2-dimensional (2D) images using computers. It also includes research on computer hardware, used for image computing and displaying. Among applications of computer graphics there are: typography, cinema, entertainment, scientific visualization and industrial design. Implicitly, the 2D images contain information about how light beams pass through the space and reach our eyes. Then a snapshot of this light is interpreted by our perception as a sight full of shapes and colors. Images in a computer are usually represented with two major archetypes:

\begin{itemize}
	\item vector images \cite{aux:vector14} - represented by a set of 2D points, lines, polygonal shapes and mathematically defined distribution of color within them;
	\item raster images \cite{aux:raster94} - represented by a uniform rectangular grid (referred to as raster), filled with colored squares (referred to as pixels). Typically, colors are defined using Red-Green-Blue (RGB) color model \cite{aux:color05}, where all possible colors are mixed from 3 primary colors, stored in computers as values in $[0;1]$ range of real numbers (or similarly as values in $[0;255]$ range of integer numbers).
\end{itemize}

Vector graphics is more practical to visualize simple shapes, like schemes or text. This representation is not suited for objects of the real world, as they are too complex: may have arbitrary geometry, specular properties of materials, when being illuminated by light or shadowed by other objects. Representing all of that with vector graphics would require hundreds of thousands of unique shapes, which is a lot in terms of computations, human hand-craft, and computer file sizes. 

Although in the raster graphics, the number of pixels may also surpass millions, the structure of pixel grids is uniform, and each pixel stores the same amount of data. Also for most cases, the image data has information redundancy in it, and thus can be compressed \cite{aux:compression18}. Besides, the pixel grid resembles a canvas from the traditional art. Current computer software for creation of raster images has capabilities similar to real brushes and palettes. This intuitive representation allows to speed up production time for humans. And we know that raster images can be very realistic, proven by the visual quality that can be captured with modern digital cameras \cite{aux:camera21}.

Speaking of realism, it's one of the most important challenges in computer graphics research. The generated images should ideally be indistinguishable from sights we perceive with our own eyes. Unfortunately, the real world is at least 3-dimensional (3D). Thus, it cannot be fully represented with images on a screen or paper, which are 2D by nature. Another obstacle, is that human vision is very sensitive, and it can easily detect when something is unrealistic, or has anomalous features, e.g. noise, tearing, blurriness and so on. Those are referred to as visual artifacts (also glitches, distortions). However, to this day there haven't been discovered a unified metric of realism, that would allow to compare two similar images and tell which will be more plausible for the human perception \cite{metric:wang11}.

Direct usage of 2D images for visualization of the 3D world isn't always efficient. For example, if we intend to visualize objects as a sequence of temporarily coherent images (animation), it would require to draw the same content with slight adjustment between images. Instead of manually copying the content from one image to another and insuring its coherency and synchronization, it's common to instead approximately model the 3D scene of interest. Such scene models can take a form of point clouds, voxel grids, triangular meshes, or even implicit functions, such as signed distance fields (SDF). After modeling, at every moment of time the appearance of the geometry can be projected from a certain view point onto an image. This allows to model the scene one time and reuse it for automatic generation of dozens of images.

%A common way to model 3D geometry is triangular meshes \cite{mesh-data-structure}. Here the geometry is defined as a set of 3D vertices and their connectivity into polygons (usually triangles).

Even with the 3D modeling, it's still a challenge to achieve photo-realism. Some 3D scene representations lack details, some may not be fast enough to process all the data, require enormous amount of computer storage, or require manual artistic work to look good. In either case an order of magnitude higher number of geometric details needs to be defined compared to just a 2D image. 

Typically 2D images are synthesized from a 3D scene, using rendering algorithms. Those have been developed to emulate the laws of physics that lie withing synthesis of image in traditional cameras. One of such algorithms is ray tracing, which mimics how rays travel starting from a light source, bouncing from reflective surfaces, until finally reaching a camera frame. This method is one of the most photo-realistic (atleast to the state of our theoretical understanding of physical specular phenomena), yet exceptionally long to compute in great details. That's why a much more coarse algorithm of rasterization is typically used, with the scene modeled as triangular meshes (sets of 3D triangles) and materials, that capture physical properties of surfaces, such as opacity, refraction, reflection, texture, etc. But still, there persists a problem that all those properties need to be explicitly specified, often by a human.

In the last decades, the methods of Artificial Intelligence (AI) and its sub-fields of Machine Learning (ML) and Deep Neural Networks (DNN) have emerged to solve multidisciplinary problems of science and business. Instead of modeling a phenomenon imperatively, they aim to approximate statistical properties of data captured from the phenomenon. Generally, input data is passed down a highly-parameterized computational graph, and output is compared to certain desired data. During a process referred to as a optimization (training, tuning), the parameters of the graph are iteratively adjusted to better fit the desired data. Afterwards, the graph is expected to be general enough to also accurately model the data, that was never seen during training. The automation is what makes these approaches groundbreaking. Unlike humans, numerical algorithms can capture very complex statistical correlations or patterns in highly-dimensional data, including images.

A selection of computer graphics problems may also be solved by means of ML and DNN methods. A few notable research topics in this field include:
\begin{itemize}
\item image impainting -- completion of missing parts on an image;
\item image super-resolution -- up-scaling an image and making it more detailed;
\item reconstruction of 3D objects -- automated creation of triangular meshes, point clouds, voxel grids, implicit functions that model the 3D objects' geometry and materials;
\item neural rendering -- automated synthesis of realistic images with certain expected content.
\end{itemize}

This thesis focuses on topic of neural rendering. This term on its own doesn't have a specific definition. At first it was understood as synthesis of images with help of ML and DNNs. A popular approach was to use traditional algorithms to generate a coarse unrealistic image that encodes supporting information about a scene, such as general outlines of each separate object, distance from a camera, etc. Then a DNN was used to generate the final image, likely much more realistic than what was input. In the last years, another branch of research stood up, where the DNNs are used to represent the 3D scene, and then this representation is decoded into an images using deterministic algorithms. Whilst producing outstanding results for novel camera views, such methods are far from being uses 
 
 Neural rendering description
 
 Neural rendering limitations
 
 Having an algorithm for generating photo-realistic images is only a half of the challenge. Every ML or DNN method is meant to be executed on some computing device. It be a stationary computer with lots of computing power. It can also be a mobile device, following the latest advances in mobile hardware, making them as comparably powerful with much smaller sizes than a computer. On mobile, many performance, compute and power usage constraints may arise:
 \begin{itemize}
 	\item  Models that run in real-time applications need to deliver results multiple times per
 	second.
 	\item  Models that run on mobile hardware have limited computing resources, which are
 	even further restricted, in order to reduce overheat and improve battery life.
 	\item  Models that utilize customer devices by any means need to deal with compatibility
 	issues (e.g. between operating systems).
 \end{itemize}

 This makes optimization of AI models for real usage a very challenging task. In the
 absence of a general way to efficiently run arbitrary AI models on every hardware, AI
 projects have to seek ways to run particular models on particular hardware. This is a
 crucial task to add business value to AI products.

 
 Mobile limitations
 
 Desire to make real-time mobile graphics, complications
 
 Examples of existing products, state
 
 Main thesis goal
 
 Samsung AI Center at Moscow had been researching on real-time generation of images
 with full-body human avatars. Their recent Deep Neural Network (DNN architecture
 features mesh-based neural rendering 2. Specifically, it's a Generative Adversarial
 Network (GAN, that produces a colorful 2D image of a person, given an intermediate
 multichannel 2D image of a generic human 3D model. Such DNN has proven to generate
 high quality images of avatars, when trained with big image resolution. However, the
 performance concerns for real time avatar generation weren't profoundly researched.
 
 During the Summer Immersion, a mobile Android application was developed from
 scratch. It serves as a test of performance, visual quality and compatibility of neural
 rendering DNNs with mobile devices. The application visualizes an avatar on top of
 camera frames in real-time. The visualization is essentially Augmented Reality (AR, i.e.
 the avatar is positioned as if it's a part of the real world. Camera movements allow
 observing the avatar from arbitrary distance and angle.
 
 The main concern now is to find performance and visual quality balance of the AR
 application on the mobile devices. In instance, with the minimal acceptable avatar
 quality, the performance of the researched DNNs is too low. Each frame computing
 takes more than 60ms on modern mobile hardware. However, twice as lower
 computation time is required to consider AR performance acceptable (which is above 30
 frames per second).

 
 The goal of this thesis is to continue the aforementioned project. The high-level task is
 to obtain an admissible trade off of performance and visual quality of neural rendering
 DNNs, when executed as a part of mobile AR experience. Target devices are the latest
 Samsung mobile phones running on Qualcomm Snapdragon System on the Chip (SoC,
 with Android operating system.
 In details, this includes the following parts:
  Research on modifications of existing DNN architecture for neural avatar rendering
 2, or new architectures. Research on distillation of those. In the end, models
 should be capable of rendering avatars with real-time performance of 30 frames
 per second and resolution of generated images above 256256 pixels.
  Research on training of the DNNs, to make them robust to abnormal inputs,
 compared to the training datasets. For example, extremely far zooms, extremely
 close zooms, rotations and angles of view.
  Achieve execution of the DNNs on specialized mobile hardware for fast low
 precision AI computations (see Development techniques section). Keep track of
 visual difference between outputs of the default models and mobile low precision
 models.
  Implement real-time pre-processing and post-processing routines for inputs and
 outputs that would compensate weaknesses of DNN models, improving either
 performance or visual quality of avatars.
  Implement real-time input generation for the neural rendering DNNs, namely:
 Evaluation of unified human body model SMPLX 6, that converts numerical
 parameters of skeleton pose into a fully fledged human 3D model.
 2D projection of the posed 3D model with a neural texture.
 
 Tasks

Expected result

%\section{Motivations for micro-optimization}


%\section{Description of micro-optimization}\label{ch1:opts}

%In order to perform a sequence of floating point operations, a normal FPU
%performs many redundant internal shifts and normalizations in the process of
%performing a sequence of operations.  However, if a compiler can
%decompose the floating point operations it needs down to the lowest level,
%it then can optimize away many of these redundant operations.  
%
%If there is some additional hardware support specifically for
%micro-optimization, there are additional optimizations that can be
%performed.  This hardware support entails extra ``guard bits'' on the
%standard floating point formats, to allow several unnormalized operations to
%be performed in a row without the loss information\footnote{A description of
%	the floating point format used is shown in figures~\ref{exponent-format}
%	and~\ref{mantissa-format}.}.  A discussion of the mathematics behind
%unnormalized arithmetic is in appendix~\ref{unnorm-math}.
%
%The optimizations that the compiler can perform fall into several categories:
%
%\subsection{Post Multiply Normalization}
%
%When more than two multiplications are performed in a row, the intermediate
%normalization of the results between multiplications can be eliminated.
%This is because with each multiplication, the mantissa can become
%denormalized by at most one bit.  If there are guard bits on the mantissas
%to prevent bits from ``falling off'' the end during multiplications, the
%normalization can be postponed until after a sequence of several
%multiplies\footnote{Using unnormalized numbers for math is not a new idea; a
%	good example of it is the Control Data CDC 6600, designed by Seymour Cray.
%	\cite{knuthwebsite} The CDC 6600 had all of its instructions performing
%	unnormalized arithmetic, with a separate {\tt NORMALIZE} instruction.}.
%
%% This is an example of how you would use tgrind to include an example
%% of source code; it is commented out in this template since the code
%% example file does not exist.  To use it, you need to remove the '%' on the
%% beginning of the line, and insert your own information in the call.
%%
%%\tagrind[htbp]{code/pmn.s.tex}{Post Multiply Normalization}{opt:pmn}
%
%As you can see, the intermediate results can be multiplied together, with no
%need for intermediate normalizations due to the guard bit.  It is only at
%the end of the operation that the normalization must be performed, in order
%to get it into a format suitable for storing in memory\footnote{Note that
%	for purposed of clarity, the pipeline delays were considered to be 0, and
%	the branches were not delayed.}.
%
%\subsection{Block Exponent}
%
%In a unoptimized sequence of additions, the sequence of operations is as
%follows for each pair of numbers ($m_1$,$e_1$) and ($m_2$,$e_2$).
%\begin{enumerate}
%	\item Compare $e_1$ and $e_2$.
%	\item Shift the mantissa associated with the smaller exponent $|e_1-e_2|$
%	places to the right.
%	\item Add $m_1$ and $m_2$.
%	\item Find the first one in the resulting mantissa.
%	\item Shift the resulting mantissa so that normalized
%	\item Adjust the exponent accordingly.
%\end{enumerate}
%
%Out of 6 steps, only one is the actual addition, and the rest are involved
%in aligning the mantissas prior to the add, and then normalizing the result
%afterward.  In the block exponent optimization, the largest mantissa is
%found to start with, and all the mantissa's shifted before any additions
%take place.  Once the mantissas have been shifted, the additions can take
%place one after another\footnote{This requires that for n consecutive
%	additions, there are $\log_{2}n$ high guard bits to prevent overflow.  In
%	the $\mu$FPU, there are 3 guard bits, making up to 8 consecutive additions
%	possible.}.  An example of the Block Exponent optimization on the expression
%X = A + B + C is given in figure~\ref{opt:be}.
%
%% This is an example of how you would use tgrind to include an example
%% of source code; it is commented out in this template since the code
%% example file does not exist.  To use it, you need to remove the '%' on the
%% beginning of the line, and insert your own information in the call.
%%
%%\tgrind[htbp]{code/be.s.tex}{Block Exponent}{opt:be}
%
%\section{Integer optimizations}
%
%As well as the floating point optimizations described above, there are
%also integer optimizations that can be used in the $\mu$FPU.  In concert
%with the floating point optimizations, these can provide a significant
%speedup.  
%
%\subsection{Conversion to fixed point}
%
%Integer operations are much faster than floating point operations; if it is
%possible to replace floating point operations with fixed point operations,
%this would provide a significant increase in speed.
%
%This conversion can either take place automatically or or based on a
%specific request from the programmer.  To do this automatically, the
%compiler must either be very smart, or play fast and loose with the accuracy
%and precision of the programmer's variables.  To be ``smart'', the computer
%must track the ranges of all the floating point variables through the
%program, and then see if there are any potential candidates for conversion
%to floating point.  This technique is discussed further in
%section~\ref{range-tracking}, where it was implemented.
%
%The other way to do this is to rely on specific hints from the programmer
%that a certain value will only assume a specific range, and that only a
%specific precision is desired.  This is somewhat more taxing on the
%programmer, in that he has to know the ranges that his values will take at
%declaration time (something normally abstracted away), but it does provide
%the opportunity for fine-tuning already working code.
%
%Potential applications of this would be simulation programs, where the
%variable represents some physical quantity; the constraints of the physical
%system may provide bounds on the range the variable can take.
%\subsection{Small Constant Multiplications}
%
%One other class of optimizations that can be done is to replace
%multiplications by small integer constants into some combination of
%additions and shifts.  Addition and shifting can be significantly faster
%than multiplication.  This is done by using some combination of
%\begin{eqnarray*}
%	a_i & = & a_j + a_k \\
%	a_i & = & 2a_j + a_k \\
%	a_i & = & 4a_j + a_k \\
%	a_i & = & 8a_j + a_k \\
%	a_i & = & a_j - a_k \\
%	a_i & = & a_j \ll m \mbox{shift}
%\end{eqnarray*}
%instead of the multiplication.  For example, to multiply $s$ by 10 and store
%the result in $r$, you could use:
%\begin{eqnarray*}
%	r & = & 4s + s\\
%	r & = & r + r
%\end{eqnarray*}
%Or by 59:
%\begin{eqnarray*}
%	t & = & 2s + s \\
%	r & = & 2t + s \\
%	r & = & 8r + t
%\end{eqnarray*}
%Similar combinations can be found for almost all of the smaller
%integers\footnote{This optimization is only an ``optimization'', of course,
%	when the amount of time spent on the shifts and adds is less than the time
%	that would be spent doing the multiplication.  Since the time costs of these
%	operations are known to the compiler in order for it to do scheduling, it is
%	easy for the compiler to determine when this optimization is worth using.}.
%\cite{einstein}
%
%\section{Other optimizations}
%
%\subsection{Low-level parallelism}
%
%The current trend is towards duplicating hardware at the lowest level to
%provide parallelism\footnote{This can been seen in the i860; floating point
%	additions and multiplications can proceed at the same time, and the RISC
%	core be moving data in and out of the floating point registers and providing
%	flow control at the same time the floating point units are active. \cite{knuthwebsite}}
%
%Conceptually, it is easy to take advantage to low-level parallelism in the
%instruction stream by simply adding more functional units to the $\mu$FPU,
%widening the instruction word to control them, and then scheduling as many
%operations to take place at one time as possible.
%
%However, simply adding more functional units can only be done so many times;
%there is only a limited amount of parallelism directly available in the
%instruction stream, and without it, much of the extra resources will go to
%waste.  One process used to make more instructions potentially schedulable
%at any given time is ``trace scheduling''.  This technique originated in the
%Bulldog compiler for the original VLIW machine, the ELI-512.
%\cite{latexcompanion,einstein}  In trace scheduling, code can be
%scheduled through many basic blocks at one time, following a single
%potential ``trace'' of program execution.  In this way, instructions that
%{\em might\/} be executed depending on a conditional branch further down in
%the instruction stream are scheduled, allowing an increase in the potential
%parallelism.  To account for the cases where the expected branch wasn't
%taken, correction code is inserted after the branches to undo the effects of
%any prematurely executed instructions.
%
%\subsection{Pipeline optimizations}
%
%In addition to having operations going on in parallel across functional
%units, it is also typical to have several operations in various stages of
%completion in each unit.  This pipelining allows the throughput of the
%functional units to be increased, with no increase in latency.
%
%There are several ways pipelined operations can be optimized.  On the
%hardware side, support can be added to allow data to be recirculated back
%into the beginning of the pipeline from the end, saving a trip through the
%registers.  On the software side, the compiler can utilize several tricks to
%try to fill up as many of the pipeline delay slots as possible, as
%seendescribed by Gibbons. \cite{knuthwebsite}


