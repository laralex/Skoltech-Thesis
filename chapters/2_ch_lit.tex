\chapter{Literature Review}\label{chapter:lit}

\section{Traditional 3D representations and rendering algorithms}
\label{lit:classic-repr}
\begin{itemize}
	\item point clouds, oriented point clouds
	\item triangular meshes
	\item voxel grids
	\item implicit functions
	\item neural representation
\end{itemize}

\alert{CPU, GPU, TPUs}
\label{lit:classic-algo}
\begin{itemize}
	\item ray casting 
	\item ray marching
	\item rasterization
\end{itemize}


\section{Neural rendering and human reconstruction}
\alert{CNN}
\label{lit:nrender}
\begin{itemize}
	\item deferred neural rendering \cite{dnn:deferred19}
	\item NeRF \cite{dnn:nerf20,dnn:phorhum22}
	\item GANs, StyleGAN \cite{dnn:gan14, dnn:stylegan-v1-19,dnn:stylegan-v2-20,dnn:stylegan-v3-21,survey:gans:18}
	\item human avatars, from video, a few shots; monocular and not \cite{dnn:volumetric-primitives21}
	\item StylePeople in details \cite{dnn:stylepeople21}
	
	Avatar rendering can be described as a learned operator $f_\theta$, that takes as an input an image with $L$ channels and spatial dimensions $H \times W$, and outputs a Red-Green-Blue-Segmentation (RGBS) colorful image with the same spatial dimensions. The input image should be a *rasterization* of a generic human triangular mesh $M$, with $L$-channels texture $T$ (also called *neural texture*), given that the image is projected using camera parameters $C$. 
	
	The idea behind this structure, is that human mesh $M$ doesn't need to carry information about fine details, e.g. face features, loose clothes and haircut. Instead, such details of a particular person's appearance can be learned in the texture $T$ and parameters of renderer $f_\theta$. 
	
	The mentioned paper proposes ways to train a GAN that generates textures $T$, and a rendering DNN $f_\theta$ from a video sequence with a person. After the training, an arbitrary rasterization of the mesh with the prepared texture $T$ can be used. For example, the mesh's skeleton can be animated, and the avatar can be observed from an arbitrary camera pose. 
	
	\item unified body models (SMPLX) \cite{dnn:smpl15, dnn:smplify16, dnn:smplx19}
	
	The human 3D model, on which the neural texture is applied, is based on work by the name SMPL-X. This unified body model is used for construction of a human's triangular mesh with more than 10 000 vertices. SMPL-X also features detailed poses of hands and faces, and provides robustness to self-intersection of joints. The input parameters to create the 3D model are:
	\begin{itemize}
		\item  Shape parameters - values that carry meaning of body scale, incorporating height, width of hips, size of a head, etc.
		\item  Joints pose - values encoding skeleton pose, location of joints, hands, fingers.
		\item  Expression parameters - values encoding facial expression.
		\item  Gender parameter - value controlling creation of male/female bodies.
	\end{itemize}

	\item training data augmentations
	
\end{itemize}

\section{Metrics of images similarity}
\label{lit:metrics}
\begin{itemize}
	\item Structural Similarity
	\item Peak Signal to Noise ratio
	\item Perceptual Similarity
	\item Image L1 distance
	\item Adversarial loss
	\item Feature matching
	\item Dice loss
	\item LPIPS
\end{itemize}

\section{Increasing computing speed of Neural Networks}
\label{lit:dnn-speedup}

State-of-the-art DNN methods indeed achieve high quality results, but may also have tremendous computational costs. As an example, the models of natural language processing may surpass billions of trained parameters (weights): Turing-NLG has 17 billion parameters \cite{dnn:turingnlg20}, GPT-3 - 175 billion \cite{dnn:gpt3-20}, Megatron-Turing NGL - 530 billion \cite{dnn:megatron22}. This number affects the amount of required computer memory, as all the parameters, computed values (activations) of intermediate DNN layers, and analytic gradients need to be stored. Secondly, sufficient computing power is needed to make training time more feasible. In instance, a computing cluster of 10000 GPUs, 285000 CPU cores was built in order to train GPT-3, still requiring more than a month of constant training \cite{aux:openai-cluster-20}. 

After training, the amount of minimum computing power to perform DNN inference is generally a lot smaller, because we can load and compute neural layers one by one without storing intermediate activations. Nevertheless, inference may still require dozens of GPUs, a handful of memory and seconds to compute. Thus, a few research directions of Deep Learning are dedicated to reduce the required computing power, trying to preserve the similar model's quality.

One such approach is known as \textit{knowledge distillation}. Its idea, is that the learned values of DNN's parameters are not equivalent to knowledge of how to perform a task. In fact, by knowledge we can consider any sufficiently good mapping from input data to the desired output data \cite{method:distillation15}. Thus, we could try to train the original model fully to yield great results, and then to train another DNN "student" model with much fewer parameters. Both models are inferred with the same input data, and the student model is supervised to output the same results as the original model. Mimicking intermediate activations of the original model is also an option to replicate even more knowledge. The simplified neural architecture of the student models helps to reduce computation time and overfitting. That's because with fewer parameters there's less capacity in the student to remember the training data, and also imperfections of the original model's predictions additionally regularize the student \cite{survey:distillation21, speed:distillgan19}. However, knowledge distillation adds up more research work, with experimentation on neural architectures of the student, the training routines and hyper-parameters.

There's an adjacent group of methods, called \textit{neural architecture search}. It aims to automatically find a DNN architecture that would outperform hand-crafted architectures on the given tasks. The motivation is to both reduce the committal effect of selecting a sub-optimal neural architecture by researchers, and to also reduce the amount of parameters required to accomplish the same tasks \cite{survey:neural-arch-search19, dnn:eff-neural-arch-search18}. However, a single search step requires to train the current network, collect quality metrics and then update the architecture. The convergence may require dozens of steps. This limits neural architecture search to lightweight DNNs, and it's hard to apply for big neural architectures.

Another option comes from an observation, that some activation values of DNNs may be less important to the overall output than the others. Such activation values usually have low variance with respect to different input data, i.e. they frequently yield similar values. The corresponding learned parameters could then be replaced in the architecture by constants, if the quality doesn't decrease much. Such approach is called \textit{pruning}. Similarly to knowledge distillation, it allows to train a big model with the full capacity, which eases training. Then the obtained knowledge is squeezed to the lower number of parameters  \cite{method:pruning16, survey:pruning20, speed:prunning-gan21}.

On a computer hardware level, DNNs are usually trained and inferred with weights and activations represented with 16-bit/32-bit floating point numbers (also referred to as half-float or FP16; full-float or FP32 respectively). A \textit{quantization} approach switches to integer number formats, with lower memory usage, e.g. 4-bit/8-bit/16-bit integer numbers (INT4/INT8/INT16). The main motivation is that integer operations can be computed faster than floating-point ones, their physical circuits are smaller, thus decreasing device's size, overheating, energy consumption \cite{aux:fp-int-speed10}. The quantization approach relies on a fact, that a vector of floating-point numbers $\bm{x}$ can be approximately expressed as 
\begin{equation}
	\setlength\abovedisplayskip{0pt}
	\bm{x} \approx s^\prime_{\bm{x}} \cdot \bm{x^\prime_{\mathtt{int}}} = s_{\bm{x}} \cdot (\bm{x_{\mathtt{int}}} - z_{\bm{x}})
	\setlength\belowdisplayskip{0pt}
\end{equation} where $s^\prime_{\bm{x}}$ is a floating-point scalar, $\bm{x^\prime_{\mathtt{int}}}$ contains only integer values. Then they can be re-parameterized using an integer zero-offset $z_{\bm{x}}$, so that values in $\bm{x_{\mathtt{int}}}$ are mapped uniformly to range $(0, 2^b-1)$, with $b$ being bit-width of a number. Having a floating-point vector $\bm{x}$, it's quantized as:  
\begin{equation}
	\setlength\abovedisplayskip{0pt}
	\bm{x_{\mathtt{int}}} = \text{round}(\bm{x}/s_{\bm{x}}) + z_{\bm{x}}.
	\setlength\belowdisplayskip{0pt}
\end{equation}

Since many operations in neural networks can be computed using dot products of vectors, this allows to represent them in a quantized form. For example, a basic operation of matrix-vector multiplication with a floating-point weights matrix $\bm{W}_{n \times m}$ and a bias vector $\bm{b}_n$, the output activation vector $\bm{A}_n$  could be computed from an input activation vector $\bm{x}_m$ as:
\setlength\abovedisplayskip{0pt}
\begin{align}
\begin{split}
	\bm{A} &= \bm{b} + \bm{W} \bm{x}\\
  &\approx \bm{b}
  + [s_{\bm{w}} (\bm{W}_{\mathtt{int}} - z_{\bm{w}})]
  \cdot [s_{\bm{x}} (\bm{x}_{\mathtt{int}} - z_{\bm{x}})] \\
  &=\bm{b}
  + s_{\bm{w}} s_{\bm{x}} ( 
  \bm{W}_{\mathtt{int}} \bm{x}_{\mathtt{int}}
  - \bm{x}_{\mathtt{int}} z_{\bm{w}}
  \textcolor{blue}{- \bm{W}_{\mathtt{int}} z_{\bm{x}}
  + z_{\bm{x}} z_{\bm{w}}}).
	\setlength\belowdisplayskip{0pt}
\end{split}
\end{align} The last two terms shown in blue can be pre-computed for the whole DNN, and merged with the bias vector $\bm{b}$ \cite{dnn:quant-white21}. However since there's still an extra overhead from computation of $\bm{x}_{\mathtt{int}} z_{\bm{w}}$, specifically for weights it's common to use symmetric quantization, where floating-point values are mapped to $(-2^{b-1}, 2^{b-1}-1)$ integer range, with floating-point 0 mapped precisely to integer 0, thus making $z_{\bm{w}} = 0$, and eliminating the overhead. 

Typically, a post training quantization (PTQ) is applied to weights and activations of a DNN, it's easy to apply and automatically get performance benefits of quantization. On the other hand, if precision of the outputs is a priority, a  quantization-aware training (QAT) \cite{quant:qat18} approach is used. Here architecture tweaks are required to do quantized inference and weights updates, with an idea to make the model learn to deal with quantization imprecision. As a downside, it enlarges training time and adds issues with analytic gradients back-propagation through rounded values \cite{quant:straight-through-estimator13}.

Usually, the uniform quantization is appealing for simplicity of hardware implementation. However, the quantization range defined by constants $s$ and $z$ may be selected too wide, e.g. min-max range of values in a tensor where outliers are present. Thus, a lot of precision will be lost as different numbers will be mapped to the same integer value. Instead, the range can be selected to minimize a mean squared error with respect to the original floating-point values \cite{quant:mse19, speed:quant-error-analysis15}. If hardware supports it, instead of defining per-tensor quantization constants, they can be defined per-channel for more precision. Alternatively, a Cross Layer Equalization \cite{quant:cle20} can be applied to re-parameterize quantized weights and biases, compensating the channels' discrepancy of ranges, without adding runtime overhead. A non-uniform quantization can also be used to devote more precision to near mean values, and less to the outliers, but the hardware support of it is very limited \cite{quant:non-uniform21}.

Moving on, DNNs also suffer from computational inefficiency, when similar input data leads to computing identical activations of layers. The DNN's architecture can be designed to cache and reuse these values. It can be applied to video stream processing or synthesis, where multiple consecutive frames have similarity \cite{aux:reusing19}. Additionally, filters of CNNs may frequently contain repeated weights after being quantized. Thus, as a filter slides over the data, certain dot-products can be reused \cite{dnn:reusing18}. While having a potential to speed up a certain DNN architecture, it requires immense manual hand-craft, which slows down active research and experimentation.

The last but not least appoach, is to design the neural architectures efficiently, i.e with a good trade off of inference speed and outputs accuracy. It's obtained by reducing the number of parameters, applying only fast-to-compute neural layers, and composing them into a sparse computational graph. It leads to a better utilization of memory caches and computing cores of the hardware. Often, such models are implemented using low-level parallelized code, that increases the number of floating-point operations per second (FLOPs) even further. The notable examples are: MobileNetV1 \cite{dnn:mnv1-17}, MobileNetV2 \cite{dnn:mnv2-18}, MobileNetV3 \cite{dnn:mnv3-19},  ShuffleNetV1 \cite{dnn:shufflenetv1-18}, ShuffleNetV2 \cite{dnn:shufflenetv2-18}, CondenseNetV1 \cite{dnn:condensenetv1-18}, CondenseNetV2 \cite{dnn:condensenetv2-21}, EfficientNet \cite{dnn:efficientnetv1-19}. Although the original papers obtain similar or better results with their proposed architectures, the drop-in replacement is not guaranteed for every task. In this work a significant drop of quality is observed, when a baseline ResNet18 backbone was replaced with either MobileNetV2 or EfficientNet.

%

\section{Mobile devices architecture}
\label{lit:mobile}

For a long time now, the computing hardware has been moving from general purpose computing units (referred to as Central Processing Units, CPU), towards specialized processing units, that are specifically constructed and programmed to perform a subset of operations faster than CPU. The most obvious example is a Graphical Processing Unit (GPU), consisting of hundreds of repetitive cores that may execute linear algebra operations in parallel. The usage of GPUs was recognized for AI purposes. Powerful full-size GPUs are the main devices to train and execute DNNs on desktop computers and clusters. On embedded hardware (for instance mobile), the technology goes further by introducing the more specialized accelerated devices, that complete certain tasks faster than mobile GPUs \cite{mobile:dl-review19}.

Different accelerated devices exist under separate names, but the similar idea - specialized computing. For example, mobile SoC by Samsung contains Neural Processing Units (NPU) for tensor computations. Google has dedicated Tensor Processing Units (TPU) for large-scale tensor computations on base of clusters of computer machines.

Qualcomm Snapdragon SoC, that is of interest for this project, may contain a Digital Signals Processor (DSP) accelerated device (besides CPU and GPU). It's a chip designed to perform low power quantized computations with built-in support to parallel execution and Single Instruction Multiple Data operations \cite{speed:online-dsp-qualcomm}.

Mobile hardware has a lot of minor differences from desktop computing hardware. One of those is that the GPU is physically located on the same chip as the CPU. Also, their memory is located on the same chip, yet mutually inaccessible in software.

Although technically hard, but accelerated devices can be programmed separately. \cite{mobile:pipelining20} suggests a pipeline approach, where a model is split into several stages. Instead of processing input from start to finish, each stage receives new input as soon as it finished the previous calculation. By combining usage of multiple accelerate devices on mobile hardware, it's possible to almost double real time performance.

 Data-free quantization \cite{speed:datafreequant19} algorithm is proposed and implemented for Qualcomm Snapdragon SoCs, and achieves near-identical outputs' quality for quantized DNNs
